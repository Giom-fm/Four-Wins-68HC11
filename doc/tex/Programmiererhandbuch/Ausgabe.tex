\newpage
\section{Ausgabe}
    In diesem Kapitel wird erläutert, wie die Ausgabe auf dem LCD realisiert wurde.
    Dabei werden die Eigenheiten des Displays und die Repräsentation von Text innerhalb des Speichers erklärt.

    \subsection{LCD}
        Als Display steht dem Board ein \textit{NHD-C12864A1Z-FSW-FBW-HTT} zur Verfügung.
        Dieses besitzt 128 x 64 Pixel. Da der Cursor jedoch 8 Pixel hoch ist,
        bildet das LCD 128 Spalten und acht Zeilen ab.

    \subsection{Spielfeld}
        Um das Spielfeld zu zentrieren wird auf jede horizontale Berechnung die Konstante 
        \textit{boardOffset} addiert.\\
        Diese berechnet sich aus folgender Formel:
        \begin{equation}
            boardOffset = \dfrac{Displaybreite - Spielfeldbreite}{2}
        \end{equation}
        Konkret bedeutet das:
        \begin{equation}
            36 = \dfrac{128 Pixel - 7 Zeilen \cdot 8 Pixel}{2}
        \end{equation}
        

    \subsection{Text}
        Da kein dynamischer Text für die Ausgabe benötigt wird, wird jeglicher Text aus dem Speicher 
        ausgelesen.
        \\
        Dabei besteht ein Buchstabe aus zwei bis vier Bytes. Zusätzlich wird ein Leerzeichen,
        in Form eines leeren Bytes, an das Ende des Buchstaben angefügt.
        \\\\
        Somit ergibt sich Beispielsweise folgende Bytefolge für den Buchstaben \textit{T}:
        \begin{center}
            \texttt{\$02,\$7E,\$02,\$00}    
        \end{center}
        Um nun eine Buchstabenfolge als Text auszugeben, wird die Länge im ersten Byte gespeichert.
        Somit ist es möglich, in einer Schleife eine Buchstabenfolge auszugeben, ohne den dahinter liegenden
        Speicher mit auszulesen. Alternativ könnte hier auch ein Stopbyte verwendet werden.
        Dies hätte den Vorteil, dass auch Texte, die länger als 254 Zeichen lang sind, Ausgegeben werden können.
        \\
        Die folgende Tabelle zeigt die Kodierung mit dem Beispieltext \textit{Player}:
        \begin{center}
            \begin{tabular}{c | l}
                P: & \texttt{30,\$7E,\$12,\$12,\$0C,\$00}\\
                l: & \texttt{\$02,\$7E,\$00} \\
                a: & \texttt{\$20,\$54,\$54,\$78,\$00}\\ 
                y: & \texttt{\$1C,\$A0,\$A0,\$7C,\$00}\\ 
                e: & \texttt{\$38,\$54,\$54,\$48,\$00}\\ 
                r: & \texttt{\$7C,\$08,\$04,\$08,\$00}\\
                 : & \texttt{\$00,\$00}
            \end{tabular}
        \end{center}
      
